\documentclass[12pt,journal,onecolumn,twoside]{IEEEtran}

\usepackage{cite}

\usepackage{algorithmic}
\usepackage{array}
%\usepackage{fixltx2e}
\usepackage{url} 

\usepackage{amsfonts}
\usepackage{amssymb}
\usepackage[cmex10]{amsmath}
\def\abs#1{\mathopen| #1 \mathclose|}


% *** Do not adjust lengths that control margins, column widths, etc. ***
% *** Do not use packages that alter fonts (such as pslatex).         ***
% There should be no need to do such things with IEEEtran.cls V1.6 and later.
% (Unless specifically asked to do so by the journal or conference you plan
% to submit to, of course. )


% correct bad hyphenation here
\hyphenation{op-tical net-works semi-conduc-tor}

\newcommand{\HRule}{\rule{\linewidth}{0.5mm}}

\begin{document}

\begin{center}  
\thispagestyle{empty}
%\includegraphics[width=0.15\textwidth]{./logo}\\[1cm]
{\Large University of Cyprus}\\
\emph{\large Electrical and Computer Engineering Department}\\
\HRule\\[0.4cm]
{\Huge  Title of PhD Thesis Proposal}\\[1cm]
\large
Name \textsc{Surname}\\[0.3cm]
\emph{\large PhD Thesis Proposal}\\[0.5cm]
\emph{Supervisor:} Prof. Name \textsc{Surname}
\vfill
\end{center}

\textbf{Abstract:} The abstract of the PhD Thesis Proposal. \\[0.5cm]

\textbf{Biography:} \textit{Name Surname} received the...
%%%%%%%%%%%%%%%%%%%%%%%%%%%%%%%%%%%%%%%%%%%%%%%%%%%%%%%%%%%%%%%%%%%%%%%%%%%%%%%%%%%%%


%
% paper title
% can use linebreaks \\ within to get better formatting as desired
\title{Title of PhD Thesis Proposal}
%
%
% author names and IEEE memberships
% note positions of commas and nonbreaking spaces ( ~ ) LaTeX will not break
% a structure at a ~ so this keeps an author's name from being broken across
% two lines.
% use \thanks{} to gain access to the first footnote area
% a separate \thanks must be used for each paragraph as LaTeX2e's \thanks
% was not built to handle multiple paragraphs
%

\author{Name Surname}

% note the % following the last \IEEEmembership and also \thanks - 
% these prevent an unwanted space from occurring between the last author name
% and the end of the author line. i.e., if you had this:
% 
% \author{....lastname \thanks{...} \thanks{...} }
%                     ^------------^------------^----Do not want these spaces!
%
% a space would be appended to the last name and could cause every name on that
% line to be shifted left slightly. This is one of those "LaTeX things". For
% instance, "\textbf{A} \textbf{B}" will typeset as "A B" not "AB". To get
% "AB" then you have to do: "\textbf{A}\textbf{B}"
% \thanks is no different in this regard, so shield the last } of each \thanks
% that ends a line with a % and do not let a space in before the next \thanks.
% Spaces after \IEEEmembership other than the last one are OK (and needed) as
% you are supposed to have spaces between the names. For what it is worth,
% this is a minor point as most people would not even notice if the said evil
% space somehow managed to creep in.



% The paper headers
\markboth{PhD Thesis Proposal}{Surname: Title of PhD Thesis Proposal}
% The only time the second header will appear is for the odd numbered pages
% after the title page when using the twoside option.
% 
% *** Note that you probably will NOT want to include the author's ***
% *** name in the headers of peer review papers.                   ***
% You can use \ifCLASSOPTIONpeerreview for conditional compilation here if
% you desire.




% use for special paper notices
\IEEEspecialpapernotice{PhD Thesis Proposal}




% make the title area
\maketitle


%\begin{abstract}
%%\boldmath
%\end{abstract}



\section{Introduction}
\thispagestyle{empty}
\IEEEPARstart{T}{he} introduction of the PhD Thesis Proposal. What is the general problem, what is the the motivation of this work, what are the general and specific objectives of this work, what is the innovation.

\section{State-of-the-art Overview}

\subsection{Research Area 1}

\subsection{Research Area 2}


\section{First Part of the Thesis}

In this section...

\subsection{Results}
The work presented in this section has been published and presented in... \cite{Surname2010}



\section{Second Part of the Thesis}

In this section...

\subsection{Results}
The work presented in this section has been published and presented in...


\section{Third Part of the Thesis}

In this section...

\subsection{Results}
The work presented in this section has been published and presented in...


\section{Fourth Part of the Thesis}

In this section...

\subsection{Results}
The work presented in this section has been published and presented in...


\section{Conclusions and Impact}



% Can use something like this to put references on a page
% by themselves when using endfloat and the captionsoff option.
\ifCLASSOPTIONcaptionsoff
  \newpage
\fi



% trigger a \newpage just before the given reference
% number - used to balance the columns on the last page
% adjust value as needed - may need to be readjusted if
% the document is modified later
%\IEEEtriggeratref{8}
% The "triggered" command can be changed if desired:
%\IEEEtriggercmd{\enlargethispage{-5in}}

% references section


\bibliographystyle{IEEEtran}

\bibliography{thesis}

% can use a bibliography generated by BibTeX as a .bbl file
% BibTeX documentation can be easily obtained at:
% http://www.ctan.org/tex-archive/biblio/bibtex/contrib/doc/
% The IEEEtran BibTeX style support page is at:
% http://www.michaelshell.org/tex/ieeetran/bibtex/
%\bibliographystyle{IEEEtran}
% argument is your BibTeX string definitions and bibliography database(s)
%\bibliography{IEEEabrv,../bib/paper}
%
% <OR> manually copy in the resultant .bbl file
% set second argument of \begin to the number of references
% (used to reserve space for the reference number labels box)
%\begin{thebibliography}{1}
%
%\bibitem{IEEEhowto:kopka}
%H.~Kopka and P.~W. Daly, \emph{A Guide to \LaTeX}, 3rd~ed.\hskip 1em plus
%  0.5em minus 0.4em\relax Harlow, England: Addison-Wesley, 1999.
%
%\end{thebibliography}

% biography section
% 
% If you have an EPS/PDF photo (graphicx package needed) extra braces are
% needed around the contents of the optional argument to biography to prevent
% the LaTeX parser from getting confused when it sees the complicated
% \includegraphics command within an optional argument. (You could create
% your own custom macro containing the \includegraphics command to make things
% simpler here.)
%\begin{biography}[{\includegraphics[width=1in,height=1.25in,clip,keepaspectratio]{mshell}}]{Michael Shell}
% or if you just want to reserve a space for a photo:

% if you will not have a photo at all:
%\begin{IEEEbiographynophoto}{Demetrios G. Eliades}
%Biography text here.
%\end{IEEEbiographynophoto}

% insert where needed to balance the two columns on the last page with
% biographies
%\newpage


% You can push biographies down or up by placing
% a \vfill before or after them. The appropriate
% use of \vfill depends on what kind of text is
% on the last page and whether or not the columns
% are being equalized.

%\vfill

% Can be used to pull up biographies so that the bottom of the last one
% is flush with the other column.
%\enlargethispage{-5in}



% that's all folks
\end{document}


